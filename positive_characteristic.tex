\section{Hecke algebras in positive characteristic}
    
    When the field $C$ of values of the representation has positive characteristic, then many of the previous results do not hold. For instance, the category $\cR_C(G)$ does not have a nice block decomposition similar to the Bernstein decomposition for complex smooth representations. As a consequence, there is no analogous theory of types for positive characteristic. Nevertheless, for the setting of depth-zero representations, a bijection is still expected between simple right $\cH(G,I,\rho_\chi)$-modules and irreducible $C$-representations $(\pi,V)$ of $G(F)$ such that the $\rho_\chi$-isotypic part of $\pi|_I$ is non-zero. 

    This last point suggests that studying the structure of $\cH(G,I,\rho_\chi)$ yields relevant information about the structure of $\cR_C(G)$. Fortunately, most steps in the proof of the theorem still hold if we replace $\CC$ for an algebraically closed field of positive characteristic other than $p$. It is still true that the elements $\varphi_w, w\in\tilde{W}_\chi$ still form a basis for $\cH(G,I,\rho_\chi)$, that $\varphi_{w_1w_2}=\varphi_{w_1}*\varphi_{w_2}$ whenever $l_\chi(w_1w_2)=l_\chi(w_1)+l_\chi(w_2)$ and that the elements $\varphi_s, s\in S_{\chi,\aff}$ form a two dimensional subalgebra inside $\cH(G,I,\chi_\rho)$. However, the calculations for the coefficients for the quadratic relations rely crucially on semisimplicity of certain representations and character theory, both of which fail for small positive characteristic. 

    The aim of this document is prove in a direct, explicit way, that the same quadratic relations hold even when the characteristic is small when $G$ is a reductive group of type $A_n$.

    The first part of the proof can be easily deduced from the following fact.

    \begin{lemma}
        Let $w\in\tilde{W}_\chi$. Then
        $$[I:I\cap wIw^{-1}]=q^{l(w)}.$$
    \end{lemma}

    \begin{lemma}
        Let $s\in S_{\chi,\aff}$. Then
        $$\varphi_s^2(1)=q.$$
    \end{lemma}
    \begin{proof}
        This is a direct computation. Indeed,
        \begin{equation*}
            \varphi_s^2(1)=\int_{IsI}\varphi_s(h)\varphi_s(h^{-1})dh=\chi(s^2)^{-1}\varphi_s(s)^2[IsI:I]=q^{1-l(s)}[I:I\cap sIs^{-1}]=q.
        \end{equation*}
        %since $s^2=\calpha(\pm1)$ for $\alpha\in\Phi_\chi$.
    \end{proof}

    Then, the hard part of the proof is to show that $\varphi_s^2(s)=(q-1)\varphi_s(s)=(q-1)q^{1-l(s)}\check{\chi}(s)$ if $s\in S_{\chi,\aff}$.
    The first step for proving this result consists in 

    \subsection{Quadratic relation for a canonical example}
    We now prove the quadratic relation for a family of reflections containing one reflection for each possible length. We remark that any $s\in S_{\chi,\aff}$ is a reflection and, as such, $l(s)$ will always be odd.

    Let $n\geq1$ be a positive integer and let $G=\mathrm{GL}_{n+1}$. Choose a set of simple roots $\Pi=\{\alpha_1,\ldots,\alpha_n\}$ with longest root $\alpha_0=\alpha_1+\cdots+\alpha_n$. In this case, the Weyl group $W=N(F)/T(F)$ is generated by $S=\{w_{\alpha_1},\ldots,w_{\alpha_n}\}$ and the extended Weyl group $\tilde{W}=N(F)/T(\cO)$ decomposes as a semidirect product of the affine Weyl group $W_{\aff}=N(F)|_{\det\in\cO^\times}/T(\cO)$ and the alcove stabilizer $\Omega=\langle\sigma\rangle\cong\ZZ,$ where
    $$\rho=\begin{pmatrix}
        0&1&&\\
        &\ddots&\ddots&\\
        &&\ddots&1\\
        -\varpi &&&0\\
    \end{pmatrix}.$$

    As discussed above, the affine Weyl group is a coxeter group, where the simple reflections are given by 
    $$S_{\aff}=\{w_{\alpha_1}(1),\ldots,w_{\alpha_n}(1),w_{\alpha_0}(\varpi^{-1})\}.$$

    Let $\chi=\chi_0\otimes\chi_1\otimes\cdots\otimes\chi_n$ be a character of the torus, where each $\chi_i$ is a depth-zero character of $\cO^\times$. Assume that $\chi_0=\chi_n$ and with all others $\chi_i$ distinct characters. Under these assumptions, 
    $$N_\chi=\begin{pmatrix}
        *&&&&\\
        &*&&&\\
        &&\ddots&&\\
        &&&*&\\
        &&&&*\\
    \end{pmatrix}\bigsqcup\begin{pmatrix}
        &&&&*\\
        &*&&&\\
        &&\ddots&&\\
        &&&*&\\
        *&&&&\\
    \end{pmatrix},$$
    and in particular

    \begin{equation*}
        S_{\chi,\aff}=\left\{w_{\alpha_0}(1)=\begin{pmatrix}
            &&&&1\\
            &1&&&\\
            &&\ddots&&\\
            &&&1&\\
            1&&&&\\
        \end{pmatrix}, w_{\alpha_0}(\varpi^{-1})=\begin{pmatrix}
            &&&&\varpi^{-1}\\
            &1&&&\\
            &&\ddots&&\\
            &&&1&\\
            -\varpi&&&&\\
        \end{pmatrix}\right\}
    \end{equation*}

    \begin{lemma}\label{lem_length}
        Under the circumstances discussed above, we have that $l(w_{\alpha_0}(\varpi^{-1}))=1$ and $l(w_{\alpha_0}(1))=2n-1$.
    \end{lemma}
    \begin{proof}
        The first part of the statement is immediate since $w_{\alpha_0}\in S_{\aff}$ is a simple reflection. For the second part, we first note that $w_{\alpha_0}(1)$ is naturally an element of the Weyl group $W$, generated by $S=\{w_{\alpha_1},\ldots,w_{\alpha_n}\}$, and therefore $l(\alpha_0(1))$ coincides with its length as an element of $W$. Thus
        $$l(w_{\alpha_0})=|\{\alpha\in\Phi^+:w_{\alpha_0}(\alpha)\in\Phi^-\}|.$$
        Now, if $\alpha\in\Phi^+$, then $w_{\alpha_0}(\alpha)=\alpha-\langle\alpha,\calpha_0\rangle\alpha_0\in\Phi^-$if and only if $\langle\alpha\calpha_0\rangle>0$. Since $\langle\alpha_i,\calpha_0\rangle=1$ if $i=1,n$ and $0$ otherwise. Consequently, 
        $$\{\alpha\in\Phi^+:w_{\alpha_0}(\alpha)\in\Phi^-\}=\{\alpha_1,\alpha_1+\alpha_2,\ldots,\alpha_0,\ldots,\alpha_{n-1}+\alpha_n,\alpha_n\},$$
        which has size $2n-1$ as desired.
    \end{proof}

    The proof of the quadratic equation for $s_0=w_{\alpha_0}(\varpi^{-1})$ is straightforward. For the sake of completeness, and as means of example, we give a full proof.
    
    Since $l(s_0)=1$, we consider $\varphi_{s_0}=[Is_0I]_{\check{\chi}}$. Then
    $$\varphi_{s_0}^2(s_0)=\int_{G(F)}\varphi_{s_0}(s_0h)\varphi_{s_0}(h^{-1})dh$$
    and the integral is zero unless $h\in Is_0I\cap s_0Is_0I$. To understand this integral, we note that for any $x\in G(F)$, there is a bijection
    \begin{align*}
        IxI/I&\longleftrightarrow I/(I\cap xIx^{-1})\\
        yxI&\longmapsto y(I\cap xIx^{-1})
    \end{align*}
    and that 
    \begin{equation*}
        I\cap s_0Is_0^{-1}=\begin{pmatrix}
            \cO^\times & \cO & \cdots & \cO & \cO\\
            \varpi\cO & \cO^\times & \ddots &  & \cO\\
            \vdots &\ddots & \ddots & \ddots & \vdots \\
            \varpi\cO & \varpi\cO & \ddots & \cO^\times & \cO\\
            \varpi^2\cO & \varpi\cO & \ldots & \varpi\cO & \cO^\times
        \end{pmatrix}.
    \end{equation*}
    Therefore, by fixing a section $t:k_F\rightarrow\cO, a\mapsto a_t$ of the quotient $\cO\rightarrow k_F$, we obtain
    \begin{equation*}
        Is_0I=\bigsqcup_{c\in k_F} u_cs_0I,\quad\text{where}\quad u_c=
        \begin{pmatrix}
            %1 & 0 & \cdots & 0 & 0\\
            %0 & 1 & \ddots &  & 0\\
            %\vdots &\ddots & \ddots & \ddots & \vdots \\
            %0 & 0 & \ddots & 1 & 0\\
            %\varpi c & 0 & \ldots & 0 & 1
            1 & & \\
            & 1 & \\
            \varpi c_t & & 1
        \end{pmatrix}.
    \end{equation*}

    A left cosset $u_cs_0I$ is contained in $s_0Is_0I$ if and only if there is some $z\in k_F$ such that $s_0u_cs_0u_zc_0\in I$, and a simple calculation shows that this is the case if and only if $c\neq 0$, in which case $z=c^{-1}$. Hence, $Is_0I\cap s_0Is_0I=\sqcup_{c\in k_F^\times}u_cs_0I$ and moreover 
    $$\iota_c:=s_0u_cs_0u_{1/c}s_0\in I\quad\text{satisfies}\quad\iota_c\equiv\begin{pmatrix}
        -c &&&& \\
        & 1 &&& \\
        && \ddots && \\
        &&& 1 & \\
        &&&& -1/c \\
    \end{pmatrix}\pmod{I^+},$$
    so $\rho_\chi(\iota_c)=\chi_0(-c)\chi_n(-1/c)=1$ since $\chi_0=\chi_n$. Consequently, 
    \begin{align*}
        \varphi_{s_0}^2(s_0)=\sum_{c\in k_F^\times}\int_{u_cs_0I}\varphi_{s_0}(sh)\varphi_{s_0}(h^{-1})dh=\sum_{c\in k_F^\times}\int_{I}\varphi_{s_0}(su_csk)\varphi_{s_0}(k^{-1}s_0^{-1}u_c^{-1})dk=\\
        =\sum_{c\in k_F^\times}\varphi_{s_0}(\iota_cs_0^{-1}u_{1/c})\varphi_{s_0}(s_0^{-1}u_c^{-1})=\varphi_{s_0}(s_0)^2\sum_{c\in k_F^\times}\rho_\chi(\iota_c)\rho_\chi(u_{1/c})^{-1}\rho_\chi(u_c)^{-1}=|k_F^\times|=q-1,
    \end{align*}
    and this completes the proof that $\varphi_{s_0}^2=(q-1)\varphi_{s_0}+q\varphi_1$.
    
    Next, we turn out attention towards proving the quadratic relation for $s=w_{\alpha_0}(1)\in S_{\chi,\aff}$. In Lemma \ref{lem_length} we proved that $l(s)=2n-1$ and that 
    $$\{\alpha\in\Phi^+:w_{\alpha_0}(\alpha)\in\Phi^-\}=\{\alpha_1,\alpha_1+\alpha_2,\ldots,\alpha_0,\ldots,\alpha_{n-1}+\alpha_n,\alpha_n\},$$
    which implies that
    \begin{equation*}
        I\cap sI s^{-1}=\begin{pmatrix}
            \cO^\times&\varpi\cO&\varpi\cO&\cdots&\varpi\cO&\varpi\cO\\
            \varpi\cO&\cO^\times&\cO&\cdots&\cO&\varpi\cO\\
            \varpi\cO&\varpi\cO&\ddots&\ddots&\vdots&\vdots\\
            \vdots&\vdots&\ddots&\ddots&\cO&\varpi\cO\\
            \varpi\cO&\varpi\cO&\ldots&\varpi\cO&\cO^\times&\varpi\cO\\
           \varpi\cO&\varpi\cO&\ldots&\varpi\cO&\varpi\cO&\cO^\times\\
        \end{pmatrix}
    \end{equation*}
    and therefore we can express the double cosset $IsI$ as a disjoint union of left cossets of $I$ as
    \begin{equation*}
        IsI=\bigsqcup_{(\mathbf{a},\mathbf{b},c)\in k_F^{2n-1}}u_{\mathbf{a},\mathbf{b},c}sI\quad\text{where}\quad u_{\mathbf{a},\mathbf{b},c}=\begin{pmatrix}
            1&a_1&a_2&\cdots&a_{n-1}&c\\
            0&1&0&\cdots&0&b_{n-1}\\
            \vdots&\ddots&\ddots&\ddots&\vdots&\vdots\\
            \vdots&&\ddots&\ddots&0&b_2\\
            \vdots&&&\ddots&1&b_1\\
            0&\ldots&\ldots&\ldots&0&1\\
        \end{pmatrix}
    \end{equation*}
    Here, we are abusing notation, and all elements of the matrix are assumed to be the lifts from $k_F$ to $\cO$ by the section $t$ fixed above.
    \begin{lemma}
        Let $(\mathbf{a},\mathbf{b},c)\in k_F^{2n-1}$. Then there is some $(\mathbf{x},\mathbf{y},z)\in k_F^{2n-1}$ such that 
        $$su_{\mathbf{a},\mathbf{b},c}su_{\mathbf{x},\mathbf{y},z}s\in I$$
        if and only if 
        \begin{equation*}
        c(c-\sum_{i+j=n}a_ib_j)\neq0\quad\text{and}\quad a_ib_j=0 \text{  if  } i+j<n.
        \end{equation*}
        When these conditions hold, then 
        $$\mathbf{x}=\mathbf{a}c^{-1},\quad \mathbf{y}=\mathbf{b}(c-\sum_{i+j=n}a_ib_j)^{-1}\quad\text{and}\quad z=(c-\sum_{i+j=n}a_ib_j)^{-1}$$
        and 
        \begin{equation}
            \iota_{\mathbf{a},\mathbf{b},c}:=su_{\mathbf{a},\mathbf{b},c}su_{\mathbf{x},\mathbf{y},z}s\in I\quad\text{satisfies}\quad \iota_{\mathbf{a},\mathbf{b},c}\equiv
            \begin{pmatrix}
                \frac{1}{c-\sum_{i+j=n}a_ib_j}&&&& \\
                &\frac{c-a_1b_{n-1}}{c}&&& \\
                &&\ddots&& \\
                &&&\frac{c-a_{n-1}b_1}{c}& \\
                &&&&c
            \end{pmatrix}
            \pmod{I^+}.
        \end{equation}
    \end{lemma}
    To simplify notation, we let $$J_n=\{(\mathbf{a},\mathbf{b},c)\in k_F^{2n-1}\ |\ c(c-\sum_{i+j=n}a_ib_j)\neq0\quad\text{and}\quad a_ib_j=0 \text{  if  } i+j<n\}.$$
    \begin{cor}\label{cor_sIsI}
        We have that
        $$IsI\cap sIsI=\bigsqcup_{(\mathbf{a},\mathbf{b},c)\in J_n}u_{\mathbf{a},\mathbf{b},c}sI$$
    \end{cor}

    Now, we let $\varphi_s=q^{(1-l(s))/2}[IsI]-{\check{\chi}}=q^{1-n}[IsI]_{\check{\chi}}$, and using the previous results, we obtain
    \begin{align*}
        \varphi_s^2(s)=\int_{IsI\cap sIsI}\varphi_s(sh)\varphi_s(h^{-1})dh=\sum_{(\mathbf{a},\mathbf{b},c)\in J_n}\int_{u_{\mathbf{a},\mathbf{b},c}sI}\varphi_s(sh)\varphi_s(h^{-1})dh=\\
        =\sum_{(\mathbf{a},\mathbf{b},c)\in J_n}\int_{I}\varphi_s(su_{\mathbf{a},\mathbf{b},c}sk)\varphi_s(k^{-1}s^{-1}u_{\mathbf{a},\mathbf{b},c}^{-1})dk=\sum_{(\mathbf{a},\mathbf{b},c)\in J_n}\varphi_s(\iota_{\mathbf{a},\mathbf{b},c}s^{-1}u_{\mathbf{x},\mathbf{y},z})\varphi_s(s^{-1}u_{\mathbf{a},\mathbf{b},c}^{-1})=\\
        =\rho_\chi(s^{-2})^2\sum_{(\mathbf{a},\mathbf{b},c)\in J_n}\rho_\chi(\iota_{\mathbf{a},\mathbf{b},c})\varphi_s(s)\rho_\chi(u_{\mathbf{x},\mathbf{y},z})^{-1}\varphi_s(s)\rho_\chi(u_{\mathbf{a},\mathbf{b},c})^{-1}=q^{2-2n}\sum_{(\mathbf{a},\mathbf{b},c)\in J_n}\rho_\chi(\iota_{\mathbf{a},\mathbf{b},c}).
    \end{align*}

    To calculate $\varphi_s^2(s)$, it is therefore enough to compute the sum
    \begin{align*}
        R_n(q,\{\chi_0,\ldots,\chi_{n-1}\})&=\sum_{(\mathbf{a},\mathbf{b},c)\in J_n}\rho_\chi(\iota_{\mathbf{a},\mathbf{b},c})=\\
        &=\sum_{(\mathbf{a},\mathbf{b},c)\in J_n}\chi_0\left(\frac{c}{c-\sum_{i+j=n}a_ib_j}\right)\chi_1\left(\frac{c-a_1b_{n-1}}{c}\right)%\chi_2\left(\frac{c-a_2b_{n-2}}{c}\right)
        \cdots\chi_{n-1}\left(\frac{c-a_{n-1}b_1}{c}\right).
    \end{align*}

    \begin{proposition}\label{prop_Rvalues}
        For $n\geq 3$ and $\chi_0\not\in\{\chi_1,\ldots,\chi_{n-1}\}$, the sums $R_n(q,\{\chi_0,\ldots,\chi_{n-1}\})$ satisfy the recurrence relation
        \begin{equation*}%\label{eqn_recrel}
            R_n(q,\{\chi_0,\ldots,\chi_{n-1}\})=qR_{n-1}(q,\{\chi_0,\chi_2,\ldots,\chi_{n-1}\})+qR_{n-1}(q,\{\chi_0,\ldots,\chi_{n-2}\})-q^2R_{n-2}(q,\{\chi_0,\chi_2,\ldots,\chi_{n-2}\}).
        \end{equation*}
        %together with $R_1(q)=q-1$ and $R_2(q)=(q-1)q$.
    \end{proposition}

    To prove this proposition, we will need the following lemma.

    \begin{lemma}\label{lem_Jsets}
        There is a bijection between the sets $\{(\mathbf{a},\mathbf{b},c)\in J_n:a_1=0\}$ and $k_F\times J_{n-1}$ given by
        $$((a_1,\ldots,a_{n-1}),(b_1,\ldots,b_{n-1}),c)\longmapsto (b_{n-1},(a_2,\ldots,a_{n-1}),(b_1,\ldots,b_{n-2}),c).$$
        Analogously, there is a bijection between the sets $\{(\mathbf{a},\mathbf{b},c)\in J_n:b_1=0\}$ and $k_F\times J_{n-1}$.
    \end{lemma}
    \begin{proof}
        This is a direct computation. If $a_1=0$, then $b_{n-1}$ is a free variable and if we let $a'_i=a_{i+1}$ for $i=1,\ldots,n-2$, then the condition $c(c-\sum_{i+j=n}a_ib_j)\neq0$ becomes $c(c-\sum_{i+j=n-1}a'_ib_j)\neq0$ and the condition $a_ib_j=0$ for $i+j<n$ becomes $a_ib_j<n-1$, as desired.
    \end{proof}

    \begin{proof}[Proof of Proposition \ref{prop_Rvalues}]
        Firstly, we decompose the sum $R_n(q,\{\chi_0,\ldots,\chi_{n-1}\})$ into
        \begin{equation}\label{eq_Rsum}
            R_n(q,\{\chi_0,\ldots,\chi_{n-1}\})=R_n^{a_1}(q,\{\chi_0,\ldots,\chi_{n-1}\})+R_n^{b_1}(q,\{\chi_0,\ldots,\chi_{n-1}\})-R_n^{a_1,b_1}(q,\{\chi_0,\ldots,\chi_{n-1}\}),
        \end{equation}
        where
        \begin{align*}
            R_n^{a_1}(q,\{\chi_0,\ldots,&\chi_{n-1}\})=\sum_{\substack{(\mathbf{a},\mathbf{b},c)\in J_n\\ a_1=0}}\rho_\chi(\iota_{\mathbf{a},\mathbf{b},c}), \quad R_n^{b_1}(q,\{\chi_0,\ldots,\chi_{n-1}\})=\sum_{\substack{(\mathbf{a},\mathbf{b},c)\in J_n\\ b_1=0}}\rho_\chi(\iota_{\mathbf{a},\mathbf{b},c})\\
            &\text{and}\quad R_n^{a_1,b_1}(q,\{\chi_0,\ldots,\chi_{n-1}\})=\sum_{\substack{(\mathbf{a},\mathbf{b},c)\in J_n\\ a_1=b_1=0}}\rho_\chi(\iota_{\mathbf{a},\mathbf{b},c}).
        \end{align*}

        Next, we can compute each individual term from \eqref{eq_Rsum}. If $a_1=0$, we note that $b_{n-1}$ is completely free, and if we let $\mathbf{a}'=(a_2,\ldots,a_{n-1})$ and $\mathbf{b}'=(b_1,\ldots,b_{n-2})$, by Lemma \ref{lem_Jsets} we obtain
        \begin{align*}
            R_n^{a_1}(q,\{\chi_0,\ldots,\chi_{n-1}\})=\sum_{\substack{(\mathbf{a},\mathbf{b},c)\in J_n\\ a_1=0}}\chi_0\left(\frac{c}{c-\sum_{i+j=n}a_ib_j}\right)\chi_2\left(\frac{c-a_2b_{n-2}}{c}\right)%\chi_2\left(\frac{c-a_2b_{n-2}}{c}\right)
            \cdots\chi_{n-1}\left(\frac{c-a_{n-1}b_1}{c}\right)=\\
            =q\sum_{(\mathbf{a}',\mathbf{b}',c)\in J_{n-1}}\chi_0\left(\frac{c}{c-\sum_{i+j=n-1}a'_ib'_j}\right)\chi_2\left(\frac{c-a'_1b'_{n-2}}{c}\right)%\chi_2\left(\frac{c-a_2b_{n-2}}{c}\right)
            \cdots\chi_{n-1}\left(\frac{c-a'_{n-2}b'_1}{c}\right)=qR_{n-1}(q,\{\chi_0,\chi_2\ldots,\chi_{n-1}\}).
        \end{align*}

        An analogous calculation shows that $R_n^{b_1}(q,\{\chi_0,\ldots,\chi_{n-1}\})=qR_{n-1}(q,,\{\chi_0,\ldots,\chi_{n-2}\})$. Finally, applying Lemma \ref{lem_Jsets} twice gives a bijection 
        \begin{align*}
            \{(\mathbf{a},\mathbf{b},c)\in J_n:a_1=b_1=0\}&\longrightarrow k_F^2\times J_{n-2}\\
            ((a_1,\ldots,a_{n-1}),(b_1,\ldots,b_{n-1}),c)&\longmapsto (a_{n-1},b_{n-1},(a_2,\ldots,a_{n-2}),(b_2,\ldots,b_{n-2}),c)
        \end{align*}
        and if we let $\mathbf{a}'=(a_2,\ldots,a_{n-2})$ and $\mathbf{b}'=(b_2,\ldots,b_{n-2})$, then
        \begin{align*}
            R_n^{a_1,b_1}(q,\{\chi_0,\ldots,\chi_{n-1}\})=\sum_{\substack{(\mathbf{a},\mathbf{b},c)\in J_n\\ a_1=b_1=0}}\chi_0\left(\frac{c}{c-\sum_{i+j=n}a_ib_j}\right)\chi_2\left(\frac{c-a_2b_{n-2}}{c}\right)%\chi_2\left(\frac{c-a_2b_{n-2}}{c}\right)
            \cdots\chi_{n-2}\left(\frac{c-a_{n-2}b_2}{c}\right)=\\
            =q^2\sum_{(\mathbf{a}',\mathbf{b}',c)\in J_{n-2}}\chi_0\left(\frac{c}{c-\sum_{i+j=n-2}a'_ib'_j}\right)\chi_2\left(\frac{c-a'_1b'_{n-3}}{c}\right)%\chi_2\left(\frac{c-a_2b_{n-2}}{c}\right)
            \cdots\chi_{n-1}\left(\frac{c-a'_{n-3}b'_1}{c}\right)=q^2R_{n-2}(q,\{\chi_0,\chi_2\ldots,\chi_{n-2}\}).
        \end{align*}
        Putting everything together yields the desired recurrence relation.
    \end{proof}

    Naturally, the next step is to compute these sums for small values of $n$ in order to use the above recurrence relation.

    \begin{proposition}
        For any two depth-zero characters $\chi_0,\chi_1$ of $\cO^\times$ such that $\chi_0\neq\chi_1$, we have that 
        \begin{equation}\label{eqn_Rinitial}
            R_1(q,\{\chi_0\})=q-1\quad\text{and}\quad R_2(q,\{\chi_0,\chi_1\})=(q-1)q.
        \end{equation}
        Moreover, the sums $R_n(q):=R_n(q,\{\chi_0,\ldots,\chi_{n-1}\})$ are independent of $\{\chi_0,\ldots,\chi_{n-1}\}$ and they satisfy $$R_n(q)=(q-1)q^{n-1}.$$
    \end{proposition}

    \begin{proof}
        The values of $R_1(q,\{\chi_0\})$ and $R_2(q,\{\chi_0,\chi_1\})$ can be obtain by direct computation. Indeed, 
        \begin{align*}
            R_1(q,\chi_0)=\sum_{c\in k_F^\times}\chi_0(c/c)=q-1,
        \end{align*}
        and
        \begin{align*}
            R_2(q,\{\chi_0&,\chi_1\})=\sum_{(a,b,c)\in J_2}\chi_0\left(\frac{c}{c-ab}\right)\chi_1\left(\frac{c-ab}{c}\right)=\sum_{\substack{(a,b,c)\in J_2\\ab=0}}\chi_0(1)\chi_1(1)+\sum_{\substack{(a,b,c)\in J_2\\ab\neq0}}\chi_0\chi_1^{-1}\left(\frac{c}{c-ab}\right)=\\
            &=(2q-1)(q-1)+(q-1)\sum_{\substack{c,d\in k_F^\times\\c\neq d}}\chi_0\chi_1^{-1}\left(\frac{c}{c-d}\right)=(2q-1)(q-1)+(q-1)^2\sum_{x\in k_F^\times\setminus\{1\}}\chi_0\chi_1^{-1}(x)=\\
            &=(2q-1)(q-1)-(q-1)^2=(q-1)q,
        \end{align*}
        as desired. This result, together with the recurrence relation from Proposition \ref{prop_Rvalues}, inductively shows that the sums $R_n(q,\{\chi_0,\ldots,\chi_{n-1}\})$ are independent of $\{\chi_0,\ldots,\chi_{n-1}\}$, which we shall simply write as $R_n(q)$. Combining Proposition \ref{prop_Rvalues} with \eqref{eqn_Rinitial}, we have that $R_n(q)$ satisfies the recurrence relation
        \begin{equation*}
            \begin{cases}
                R_n(q)=2qR_{n-1}(q)-q^2R_{n-2}(q) \quad\text{ for }n\geq 3,\\
                R_1(q)=q-1\quad\text{and}\quad R_2(q)=(q-1)q,
            \end{cases}
        \end{equation*}
        whose solution is $R_n(q)=(q-1)q^{n-1}$.
    \end{proof}
     
    \begin{remark}
        From Corollary \ref{cor_sIsI}, one can see that $P_n(q):=[IsI\cap sIsI:I]=|J_n|$ is a constant of interest directly related to $R_n(q)$. In fact, following a similar argument as above, one can show that the sequence $P_n(q)$ satisfies
        \begin{equation*}
            \begin{cases}
                P_n(q)=2qR_{n-1}(q)-q^2R_{n-2}(q) \quad\text{ for }n\geq 3,\\
                P_1(q)=q-1\quad\text{and}\quad P_2(q)=(q-1)(q^2-q+1),
            \end{cases}
        \end{equation*}
        whose solution is $P_n(q)=q^{n-2}(q-1)((n-1)q^2-(2n-3)q+(n-1))$.
    \end{remark}

    Now the proof of the quadratic relation for $s=w_{\alpha_0}(1)$ is immediate. Indeed,
    $$\varphi_s^2(s)=q^{2-2n}R_n(q)=(q-1)q^{1-n}=(q-1)\varphi_s(s),$$
    which implies that $$\varphi_s^2=(q-1)\varphi_s+q\varphi_1,$$
    as desired.
    \newpage
