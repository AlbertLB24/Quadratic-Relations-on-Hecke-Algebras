\section{Bernstein blocks, types and Hecke algebras}
    The Bernstein decomposition states that to understand $\cR(G)$, it is enough to understand $\cR_{[M,\sigma]}(G)$ for each $[M,\sigma]\in\mathfrak{J}(G)$. One can then understand each block using the theory of \textit{types}. 
    \begin{definition}
        Let $[M,\sigma]\in\mathfrak{J}(G)$. A pair $(K,\rho)$ consisting of a compact open subgroup $K$ of $G(F)$ and an irreducible smooth representation $\rho$ of $K$ is an $[M,\sigma]$-type if for any $(\pi,V)\in \cR_\CC(G)$, the following are equivalent:
        \begin{enumerate}
            \item $(\pi,V)\in\cR_{[M,\sigma]}(G).$
            \item $\Hom_K(\rho,\pi|_K)\neq\{0\}$; that is, $\rho\subseteq \pi|_K$.
        \end{enumerate}
    \end{definition}

    \begin{theorem}
        If $(K,\rho)$ is an $[M,\sigma]$-type, then the functor 
        \begin{align*}
            \cR_{[M,\sigma]}(G)&\longrightarrow \End_{G(F)}(\cInd_K^{G(F)}\rho)^{op}-\textrm{mod}\\
            (\pi,V)&\longmapsto\Hom_K(\rho,\pi|_K)
        \end{align*}
        is an equivalence of categories between the block $\cR_{[M,\sigma]}(G)$ and the category of right $\End_{G(F)}(\cInd_K^{G(F)}\rho)$-modules. 
    \end{theorem}
    We note that by Frobenius reciprocity, $$\Hom_K(\rho,\pi|_K)\cong\Hom_{G(F)}(\cInd_K^{G(F)}\rho,\pi),$$ so the former is naturally a right $\End_{G(F)}(\cInd_K^{G(F)}\rho)$-module, since it acts on the latter by precomposition.

    The idea of the proof is to show that if $(K,\rho)$ is a $[M,\sigma]$-type, then $\cInd_K^{G(F)}\rho$ is a projective generator of 
    $\cR_{[M,\sigma]}(G)$, and that the block is equivalent to the full subcategory $\cR_\rho(G)$ whose objects are representations $(\pi,V)$ satisfying $V=V[\rho]$, where $V[\rho]$ is the $G$-submodule of $V$ generated by its $\rho$-isotypical component $V^\rho$. The result then follows by the Morita equivalence. 

    In addition, the Hecke algebra $\End_{G(F)}(\cInd_K^{G(F)}\rho)$ can be described alternatively as a space of functions on $G$ valued on $\End_\CC(W)$, where $W$ is the vector space upon which $\rho$ acts.

    Let $\cH(G,K,\rho)$ denote the space of compactly supported functions $$\varphi:G\longrightarrow\End_\CC(W)$$
    satisfying
    $$\varphi(k_1 g k_2)=\rho(k_1)\circ\varphi(g)\circ\rho(k_2)$$
    for all $k_1,k_2\in K$ and $g\in G(F)$, equipped with a binary operation called \textit{convolution} as follows. If $\varphi_1,\varphi_2\in\cH(G,K,\rho)$, we define the function $\varphi_1*\varphi_2\in\cH(G,K,\rho)$ as
    $$\varphi_1*\varphi_2(g)=\sum_{h\in G(F)/K}\varphi_1(h)\varphi_2(h^{-1}g)=\int_{G(F)}\varphi_1(h)\varphi_2(h^{-1}g)dh,$$
    for $g\in G(F)$, where $dh$ denotes a Haar measure on $G(F)$ chosen to have measure $1$ on $K$. This operation gives $\cH(G,K,\rho)$ the structure of an associative $\CC$-algebra, whose identity element is the function $e:G\to\End_\CC(W)$ supported on $K$ such that $e(k)=\rho(k)$ for all $k\in K$.

    \begin{proposition}
        There are mutually inverse $\CC$-algebra homomorphisms
        \begin{equation}
            \Phi\longmapsto \varphi_\Phi\ :\ \End_{G(F)}(\cInd_K^{G(F)}\rho)\longrightarrow\cH(G,K,\rho)\ :\ \varphi \longmapsto \Phi_\varphi
        \end{equation}
        where
        \begin{align*}
            &\varphi_\Phi(g)(w)=\Phi(e_w)(g)\quad & (g\in G(F), w\in W)\\
            &\Phi_\varphi(f)(g)=\int_{G(F)}\varphi(h)(f(h^{-1}g))dh \quad &(f\in\cInd_K^{G(F)}\rho, g\in G(F)).
        \end{align*}
        Here, the Haar measure $dh$ is chosen so that it has measure $1$ on $K$ and $e_w\in\cInd_K^{G(F)}\rho$ is defined by
        \begin{align*}
            e_w(g)=
            \begin{cases}
                \rho(g)(w) &\text{ if } g\in K,\\
                0 &\text{ otherwise.}.
            \end{cases}
        \end{align*}        
    \end{proposition}
    \begin{proof}
        Some of the important steps:
        \begin{itemize}
            \item The maps are inverses of each other: This follows from 
            $$\Phi_\varphi(e_w)(g)=\int_{gK}\varphi(h)(e_w(h^{-1}g))dh=\int_K\varphi(gk)(e_w(k^{-1}))dk=\int_K\varphi(g)\circ\rho(k)(\rho(k)^{-1}w)=\varphi(g)(w)$$
            and 
            $$\int_{G(F)}\varphi_\Phi(h)(e_w(h^{-1}g))dh=\int_{K}\varphi_\Phi(gk)(e_w(k^{-1}))dk=\int_{K}\varphi_\Phi(g)\circ\rho(k)(\rho(k)^{-1}w)dk=\varphi_\Phi(g)(w)=\Phi(e_w)(g).$$
        \end{itemize}
    \end{proof}
