\section{Smooth representations and the Bernstein decomposition}

    Throughout this document, we let $F$ denote an arbitrary $p$-adic field with ring of integers $\mathcal{O}$, and residue field $k_F$ of characteristic $p>0$. Let $q=p^m$ be the cardinality of $k_F$ and let $\varpi$ be a uniformizer.

    Let $G$ denote a connected reductive group, defined and split over $\mathcal{O}$. The group $G(F)$ has a natural locally profinite topology coming from $F$. Fix an $F$-split
    maximal torus $T$ and a Borel subgroup $B$ containing $T$; assume $T$ and $B$ are defined over $\mathcal{O}$. Let $T_0=T(\mathcal{O})$ denote the maximal compact subgroup of $T(F)$. Let $\Phi\subset X^*(T)$ resp. $\Phi^\vee \subset X_*(T)$ denote the set of roots resp. coroots for $G$, $T$. Let $U$ resp. $\bar{U}$ denote the
    unipotent radical of $B$ resp. the Borel subgroup $\bar B \subset T$ opposite to $B$.

    %Let $\mathcal{B}(G,F)$ be the building of $G$ and let $\mathcal{A}(G,T,F)=X_*(T)\otimes\RR\subseteq \mathcal{B}(G,F)$ be the apartment corresponding to $T$, which comes equipped with a hyperplane structure arising from the affine roots. Let 
    %$$I=\langle T(\mathcal{\mathcal{O}}); \mathfrak{X}_\alpha(\mathcal{O}),\mathfrak{X}_{-\alpha}(\varpi\mathcal{O})\ |\ \alpha\in\Phi^+\rangle$$
    %be the Iwahori subgroup of $G$, and let $\textbf{a} \subset\mathcal{A}(G,T,F)$ be the unique alcove such that $G_{x,0}=I$ (first element in the Moy--Prasad filtration) for all $x\in\bf a$.

    Let $C$ be an algebraically closed field of characteristic $\ell\neq p$ and let $\cR_{C}(G)$ be the category smooth $C$-representations of $G$. That is, the objects of $\cR_{C}(G)$ are group homomorphisms $\pi: G\rightarrow GL(V)$ where $V$ is a $C$-vector space such that $\mathrm{Stab}_G(v)$ is an open subgroup of $G$ for all $v\in V$ (with respect to the locally profinite topology induced from $F$). If $V$ is finite dimensional, then the last condition is equivalent to $\ker\pi$ being open in $G(F)$. 

    If $C=\CC$, then the category $\cR_\CC(G)$ has a canonical decomposition. To state it, we consider the set of \textit{supercuspidal pairs} $(M,\sigma)$ where $M$ is an $F$-Levi subgroup of $G$ and $\sigma$ is a supercuspidal representation of $G$. 
    We say that two supercuspidal pairs $(M_1,\sigma_1)$ and $(M_2,\sigma_2)$ are \textit{inertially equivalent} if there exists some $g\in G(F)$ and \textit{unramified character} $\chi$ of $M_2$ such that $gM_1g^{-1}=M_2$ and $\prescript{g}{}{\sigma_1}\otimes\chi\cong \sigma_2$ as representations of $M_2$. We denote by $[M,\sigma]_G$ the conjugacy class of the pair $(M,\sigma)$, and we note that it is a union of $G$-conjugacy classes of supercuspidal pairs. Finally, let $\mathfrak{J}(G)=\{(M,\sigma)\text{ supercuspidal}\}/\sim$, where $\sim$ is inertial equivalence.

    If $(\pi,V)\in R_\CC(G)$ is irreducible, then there is some supercuspidal pair $(M,\sigma)$, unique up to $G$-conjugacy, such that $\pi$ is a subquotient of $i_P^G(\sigma)$, where $P=N\rtimes M$ is the unique parabolic subgroup with Levi factor $M$ and $i_P^G$ is the normalized parabolic induction. We call the $G$-conjugacy class $(M,\sigma)_G$ the supercuspidal support of $(\pi,V)$. For each $[M,\sigma]_G\in\mathfrak{J}(G)$, we define $\cR_{[M,\sigma]}(G)$ to be the full subcategory of $\cR_\CC(G)$ whose objects are smooth representations $(\pi,V)$ such that for all irreducible subquotients $\pi'$ of $\pi$, the supercuspidal support of $\pi'$ lies in $[M,\sigma]_G$.

    \begin{theorem}[Bernstein Decomposition]
        The category $\cR_\CC(G)$ has a block decomposition into full subcategories
        $$\cR_\CC(G)=\prod_{[M,\sigma]\in\mathfrak{J}(G)}\cR_{[M,\sigma]}(G).$$
    \end{theorem}
    
    \begin{remark}
        In any abstract category of representations $\cR(G)$ of a group $G$ with certain properties (smoothness, for instance) such that the category is closed under direct sums and extensions, one can always decompose $\cR(G)$ into maximal blocks as follows:

        Firstly, we consider an equivalence relation on irreducible representations of $\cR(G)$ generated by the rule that two irreducible representations $(\pi_1,V_1)$ and $(\pi_2,V_2)$ are related if there is some representation $(\sigma,W)\in\cR(G)$ and non-split short exact sequence
        $$0\longrightarrow (\pi_1,V_1)\longrightarrow(\sigma,W)\longrightarrow(\pi_2,V_2)\longrightarrow 0.$$
        For any irreducible $(\pi,V)$ we let $\cR_{[\pi,V]}(G)$ to be the full subcategory of representations such that all simple Jordan-Holder factors are equivalent to $(\pi,V)$ under $\sim$.
        \begin{theorem}
            The category $\cR(G)$ admits a block decomposition into full subcategories 
            $$\cR(G)=\prod_{[\pi,V]}\cR_{[\pi,V]}(G)$$
        \end{theorem}
        The Theorem is direct consequence of the following result.
        \begin{proposition}
            Let $(\sigma,W)\in\cR_{[\pi,V]}(G)$ and $(\sigma',W')\in\cR_{[\pi',V']}(G)$. If $(\pi,V)\not\sim(\pi',V')$, then there are no non-trivial extensions of $(\sigma',W')$ by $(\sigma,W)$. 
        \end{proposition}
        \begin{proof}
            To be written, check the notebook.
        \end{proof}
    \end{remark}
