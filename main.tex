\documentclass{article}

%%%%%%%%%%%%%%%%%%%%%%%%%%%%%%%%%%%%%%%%%%%%%%%%%%%%%%%%%%%%%%%%%%%%%%%%%%%%%%%%
%% package setup
%%%%%%%%%%%%%%%%%%%%%%%%%%%%%%%%%%%%%%%%%%%%%%%%%%%%%%%%%%%%%%%%%%%%%%%%%%%%%%%%


\usepackage[shortlabels]{enumitem}
\usepackage{amsfonts,amsmath, soul, matlab-prettifier, bm, amsthm,enumitem,amssymb,multirow,float,mathtools,bbm,array,varwidth,hyperref,bm}
\usepackage[all]{xy}
\usepackage[margin=0.9in]{geometry}
\usepackage{graphicx}
\usepackage{mathtools, caption, dsfont, tikz-cd}
\usepackage{wrapfig}

% tikzpicture

\usepackage{pgfplots}
\pgfplotsset{compat=1.15}
\usepackage{mathrsfs}
\usetikzlibrary{arrows}


\mathchardef\mhyphen="2D


\raggedbottom

%%%%%%%%%%%%%%%%%%%%%%%%%%%%%%%%%%%%%%%%%%%%%%%%%%%%%%%%%%%%%%%%%%%%%%%%%%%%%%%%
%% operators and symbols
%%%%%%%%%%%%%%%%%%%%%%%%%%%%%%%%%%%%%%%%%%%%%%%%%%%%%%%%%%%%%%%%%%%%%%%%%%%%%%%%

% operators
\newcommand{\Hom}{\mathrm{Hom}}
\newcommand{\Ext}{\mathrm{Ext}}
\newcommand{\Ker}{\mathrm{Ker}}
\newcommand{\Pic}{\mathrm{Pic}}
\newcommand{\comp}{\mathrm{comp}}
\newcommand{\Proj}{\mathrm{Proj}}
\newcommand{\rk}{\mathrm{rk}}
\newcommand{\Spec}{\mathrm{Spec}}
\newcommand{\Sym}{\mathrm{Sym}}
\newcommand{\Frob}{\mathrm{Frob}}
\newcommand{\Gal}{\mathrm{Gal}}
\newcommand{\GL}{\mathrm{GL}}
\newcommand{\SL}{\mathrm{SL}}
\newcommand{\Ind}{\mathrm{Ind}}
\newcommand{\Rep}{\mathrm{Rep}}
\newcommand{\Aut}{\mathrm{Aut}}
\newcommand{\Res}{\mathrm{Res}}
\newcommand{\Smo}{\mathrm{Smo}}
\newcommand{\Span}{\mathrm{Span}}
\newcommand{\Frac}{\mathrm{Frac}}
\newcommand{\supp}{\mathrm{supp}}
\newcommand{\End}{\mathrm{End}}
\newcommand{\St}{\mathrm{St}}
\newcommand{\Top}{\mathrm{Top}}
\newcommand{\Ab}{\mathrm{Ab}}
\newcommand{\Set}{\mathrm{Set}}
\newcommand{\ad}{\mathrm{ad}}
\newcommand{\hht}{\mathrm{ht}}

\newcommand{\PGL}{\mathrm{PGL}}
\newcommand{\PSL}{\mathrm{PSL}}
\newcommand{\Psh}{\mathrm{Psh}}
\newcommand{\Sh}{\mathrm{Sh}}
\newcommand{\Id}{\mathrm{Id}}

% greek 
\newcommand{\calpha}{\check{\alpha}}
\newcommand{\cPhi}{\check{\Phi}}
\newcommand{\cbeta}{\check{\beta}}

% shortcuts
\newcommand{\slii}{\mathfrak{sl}_2}
\newcommand{\sliii}{\mathfrak{sl}_3}
\newcommand{\gl}{\mathfrak{gl}}
\newcommand{\eW}{\widetilde{W_a}}


% mathcal
\newcommand{\cO}{\mathcal{O}}
\newcommand{\cR}{\mathcal{R}}
\newcommand{\cH}{\mathcal{H}}

% mathbb
\newcommand{\CC}{\mathbb{C}}
\newcommand{\FF}{\mathbb{F}}
\newcommand{\NN}{\mathbb{N}}
\newcommand{\PP}{\mathbb{P}}
\newcommand{\QQ}{\mathbb{Q}}
\newcommand{\RR}{\mathbb{R}}
\newcommand{\ZZ}{\mathbb{Z}}
\newcommand{\GG}{\mathbb{G}}
\newcommand{\adele}{\mathbb{A}}
\newcommand{\pp}{\mathfrak{p}}
\newcommand{\nn}{\mathfrak{n}}
\newcommand{\mm}{\mathfrak{m}}

% mathfrak
\newcommand{\fg}{\mathfrak{g}}
\newcommand{\fh}{\mathfrak{h}}
\newcommand{\fm}{\mathfrak{m}}

% shortcuts
\newcommand{\CG}{C_c^{\infty}(G)}
\newcommand{\cInd}{c\mhyphen\mathrm{Ind}}

\newcommand{\norm}[1]{\left\lVert#1\right\rVert}
\newcommand{\hatv}[1]{\overset{\vee}{\mathstrut#1}}

\DeclareMathOperator{\Ima}{Im}

\linespread{1.5}

\theoremstyle{plain}
\newtheorem{theorem}{Theorem}[section]
\newtheorem{question}[theorem]{Question}
\newtheorem{proposition}[theorem]{Proposition}
\newtheorem{convention}[theorem]{Convention}
\newtheorem{lemma}[theorem]{Lemma}
\newtheorem{cor}[theorem]{Corollary}
\newtheorem{algo}[theorem]{Algorithm}
\theoremstyle{definition}
\newtheorem{definition}[theorem]{Definition}
\newtheorem{notn}[theorem]{Notation}
\newtheorem{remark}[theorem]{Remark}
\newtheorem{example}[theorem]{Example}
\newtheorem{examples}[theorem]{Examples}
\newtheorem{fact}[theorem]{Fact}
\newtheorem*{hypothesis}{Hypothesis}
\newtheorem*{exercise}{Exercise}


\title{Quadratic Relations in Hecke Algebras}
\author{Albert Lopez Bruch}

\begin{document}
	\maketitle
	\pagenumbering{arabic}
    \section{Smooth representations and the Bernstein decomposition}

    Throughout this document, we let $F$ denote an arbitrary $p$-adic field with ring of integers $\mathcal{O}$, and residue field $k_F$ of characteristic $p>0$. Let $q=p^m$ be the cardinality of $k_F$ and let $\varpi$ be a uniformizer.

    Let $G$ denote a connected reductive group, defined and split over $\mathcal{O}$. The group $G(F)$ has a natural locally profinite topology coming from $F$. Fix an $F$-split
    maximal torus $T$ and a Borel subgroup $B$ containing $T$; assume $T$ and $B$ are defined over $\mathcal{O}$. Let $T_0=T(\mathcal{O})$ denote the maximal compact subgroup of $T(F)$. Let $\Phi\subset X^*(T)$ resp. $\check\Phi \subset X_*(T)$ denote the set of roots resp. coroots for $G$, $T$. Let $U$ resp. $\bar{U}$ denote the
    unipotent radical of $B$ resp. the Borel subgroup $\bar B \subset T$ opposite to $B$.

    Let $\mathcal{B}(G,F)$ be the building of $G$ and let $\mathcal{A}(G,T,F)$ be the apartment corresponding to $T$, which comes equipped with a hyperplane structure arising from the affine roots. Let 
    $$I=\langle T(\mathcal{\mathcal{O}}); \mathfrak{X}_\alpha(\mathcal{O}),\mathfrak{X}_{-\alpha}(\varpi\mathcal{O})\ |\ \alpha\in\Phi^+\rangle$$
    be the Iwahori subgroup of $G$, and let $\textbf{a} \subset\mathcal{A}(G,T,F)$ be the unique alcove such that $G_{x,0}=I$ (first element in the Moy--Prasad filtration) for all $x\in\bf a$.

    Let $C$ be an algebraically closed field of characteristic $\ell\neq p$ and let $\cR_{C}(G)$ be the category smooth $C$-representations of $G$. That is, the objects of $\cR_{C}(G)$ are group homomorphisms $\pi: G\rightarrow GL(V)$ where $V$ is a $C$-vector space such that $\mathrm{Stab}_G(v)$ is an open subgroup of $G$ for all $v\in V$ (with respect to the locally profinite topology induced from $F$). If $V$ is finite dimensional, then the last condition is equivalent to $\ker\pi$ being open in $G(F)$. 

    If $C=\CC$, then the category $\cR_\CC(G)$ has a canonical decomposition. To state it, we consider the set of \textit{supercuspidal pairs} $(M,\sigma)$ where $M$ is an $F$-Levi subgroup of $G$ and $\sigma$ is a supercuspidal representation of $G$. 
    We say that two supercuspidal pairs $(M_1,\sigma_1)$ and $(M_2,\sigma_2)$ are \textit{inertially equivalent} if there exists some $g\in G(F)$ and \textit{unramified character} $\chi$ of $M_2$ such that $gM_1g^{-1}=M_2$ and $\prescript{g}{}{\sigma_1}\otimes\chi\cong \sigma_2$ as representations of $M_2$. We denote by $[M,\sigma]_G$ the conjugacy class of the pair $(M,\sigma)$, and we note that it is a union of $G$-conjugacy classes of supercuspidal pairs. Finally, let $\mathfrak{J}(G)=\{(M,\sigma)\text{ supercuspidal}\}/\sim$, where $\sim$ is inertial equivalence.

    If $(\pi,V)\in R_\CC(G)$ is irreducible, then there is some supercuspidal pair $(M,\sigma)$, unique up to $G$-conjugacy, such that $\pi$ is a subquotient of $i_P^G(\sigma)$, where $P=N\rtimes M$ is the unique parabolic subgroup with Levi factor $M$ and $i_P^G$ is the normalized parabolic induction. We call the $G$-conjugacy class $(M,\sigma)_G$ the supercuspidal support of $(\pi,V)$. For each $[M,\sigma]_G\in\mathfrak{J}(G)$, we define $\cR_{[M,\sigma]}(G)$ to be the full subcategory of $\cR_\CC(G)$ whose objects are smooth representations $(\pi,V)$ such that for all irreducible subquotients $\pi'$ of $\pi$, the supercuspidal support of $\pi'$ lies in $[M,\sigma]_G$.

    \begin{theorem}[Bernstein Decomposition]
        The category $\cR_\CC(G)$ has a block decomposition into full subcategories
        $$\cR_\CC(G)=\prod_{[M,\sigma]\in\mathfrak{J}(G)}\cR_{[M,\sigma]}(G).$$
    \end{theorem}
    
    \begin{remark}
        In any abstract category of representations $\cR(G)$ of a group $G$ with certain properties (smoothness, for instance) such that the category is closed under direct sums and extensions, one can always decompose $\cR(G)$ into maximal blocks as follows:

        Firstly, we consider an equivalence relation on irreducible representations of $\cR(G)$ generated by the rule that two irreducible representations $(\pi_1,V_1)$ and $(\pi_2,V_2)$ are related if there is some representation $(\sigma,W)\in\cR(G)$ and non-split short exact sequence
        $$0\longrightarrow (\pi_1,V_1)\longrightarrow(\sigma,W)\longrightarrow(\pi_2,V_2)\longrightarrow 0.$$
        For any irreducible $(\pi,V)$ we let $\cR_{[\pi,V]}(G)$ to be the full subcategory of representations such that all simple Jordan-Holder factors are equivalent to $(\pi,V)$ under $\sim$.
        \begin{theorem}
            The category $\cR(G)$ admits a block decomposition into full subcategories 
            $$\cR(G)=\prod_{[\pi,V]}\cR_{[\pi,V]}(G)$$
        \end{theorem}
        The Theorem is direct consequence of the following result.
        \begin{proposition}
            Let $(\sigma,W)\in\cR_{[\pi,V]}(G)$ and $(\sigma',W')\in\cR_{[\pi',V']}(G)$. If $(\pi,V)\not\sim(\pi',V')$, then there are no non-trivial extensions of $(\sigma',W')$ by $(\sigma,W)$. 
        \end{proposition}
        \begin{proof}
            To be written, check the notebook.
        \end{proof}
    \end{remark}


    \section{Bernstein blocks, types and Hecke algebras}
    The Bernstein decomposition states that to understand $\cR(G)$, it is enough to understand $\cR_{[M,\sigma]}(G)$ for each $[M,\sigma]\in\mathfrak{J}(G)$. One can then understand each block using the theory of \textit{types}. 
    \begin{definition}
        Let $[M,\sigma]\in\mathfrak{J}(G)$. A pair $(K,\rho)$ consisting of a compact open subgroup $K$ of $G(F)$ and an irreducible smooth representation $\rho$ of $K$ is an $[M,\sigma]$-type if for any $(\pi,V)\in \cR_\CC(G)$, the following are equivalent:
        \begin{enumerate}
            \item $(\pi,V)\in\cR_{[M,\sigma]}(G).$
            \item $\Hom_K(\rho,\pi|_K)\neq\{0\}$; that is, $\rho\subseteq \pi|_K$.
        \end{enumerate}
    \end{definition}

    \begin{theorem}
        If $(K,\rho)$ is an $[M,\sigma]$-type, then the functor 
        \begin{align*}
            \cR_{[M,\sigma]}(G)&\longrightarrow \End_{G(F)}(\cInd_K^{G(F)}\rho)^{op}-\textrm{mod}\\
            (\pi,V)&\longmapsto\Hom_K(\rho,\pi|_K)
        \end{align*}
        is an equivalence of categories between the block $\cR_{[M,\sigma]}(G)$ and the category of right $\End_{G(F)}(\cInd_K^{G(F)}\rho)$-modules. 
    \end{theorem}
    We note that by Frobenius reciprocity, $$\Hom_K(\rho,\pi|_K)\cong\Hom_{G(F)}(\cInd_K^{G(F)}\rho,\pi),$$ so the former is naturally a right $\End_{G(F)}(\cInd_K^{G(F)}\rho)$-module, since it acts on the latter by precomposition.

    The idea of the proof is to show that if $(K,\rho)$ is a $[M,\sigma]$-type, then $\cInd_K^{G(F)}\rho$ is a projective generator of 
    $\cR_{[M,\sigma]}(G)$, and that the block is equivalent to the full subcategory $\cR_\rho(G)$ whose objects are representations $(\pi,V)$ satisfying $V=V[\rho]$, where $V[\rho]$ is the $G$-submodule of $V$ generated by its $\rho$-isotypical component $V^\rho$. The result then follows by the Morita equivalence. 

    In addition, the Hecke algebra $\End_{G(F)}(\cInd_K^{G(F)}\rho)$ can be described alternatively as a space of functions on $G$ valued on $\End_\CC(W)$, where $W$ is the vector space upon which $\rho$ acts.

    Let $\cH(G,K,\rho)$ denote the space of compactly supported functions $$\varphi:G\longrightarrow\End_\CC(W)$$
    satisfying
    $$\varphi(k_1 g k_2)=\rho(k_1)\circ\varphi(g)\circ\rho(k_2)$$
    for all $k_1,k_2\in K$ and $g\in G(F)$, equipped with a binary operation called \textit{convolution} as follows. If $\varphi_1,\varphi_2\in\cH(G,K,\rho)$, we define the function $\varphi_1*\varphi_2\in\cH(G,K,\rho)$ as
    $$\varphi_1*\varphi_2(g)=\sum_{h\in G(F)/K}\varphi_1(h)\varphi_2(h^{-1}g)=\int_{G(F)}\varphi_1(h)\varphi_2(h^{-1}g)dh,$$
    for $g\in G(F)$, where $dh$ denotes a Haar measure on $G(F)$ chosen to have measure $1$ on $K$. This operation gives $\cH(G,K,\rho)$ the structure of an associative $\CC$-algebra, whose identity element is the function $e:G\to\End_\CC(W)$ supported on $K$ such that $e(k)=\rho(k)$ for all $k\in K$.

    \begin{proposition}
        There are mutually inverse $\CC$-algebra homomorphisms
        \begin{equation}
            \Phi\longmapsto \varphi_\Phi\ :\ \End_{G(F)}(\cInd_K^{G(F)}\rho)\longrightarrow\cH(G,K,\rho)\ :\ \varphi \longmapsto \Phi_\varphi
        \end{equation}
        where
        \begin{align*}
            &\varphi_\Phi(g)(w)=\Phi(f_w)(g)\quad & (g\in G(F), w\in W)\\
            &\Phi_\varphi(f)(g)=\int_{G(F)}\varphi(h)(f(h^{-1}g))dh \quad &(f\in\cInd_K^{G(F)}\rho, g\in G(F)).
        \end{align*}
        Here, $f_w\in\cInd_K^{G(F)}\rho$ is defined by
        \begin{align*}
            f_w(g)=
            \begin{cases}
                \rho(g)(w) &\text{ if } g\in K,\\
                0 &\text{ otherwise.}.
            \end{cases}
        \end{align*}        
    \end{proposition}
    \begin{proof}
        Long routine checks.
    \end{proof}

    \section{Depth-zero Hecke algebras for principal series blocks}
    Let $\cR_{[M,\sigma]}(G)$ be a Bernstein block. If $M$ is the maximal torus $T$ (or a conjugate of $T$), then $\sigma=\chi$ is just a character and the $\cR_{[T,\chi]}(G)$ is called a principal series block. These are the ones we aim to understand. 

    We remark that the block is completely determined the values of $\chi$ on $T(\cO)$, which has a natural filtration of open compact subgroups
    $$T(\cO)\supset T(1+\varpi\cO)\supset T(1+\varpi^2\cO)\supset T(1+\varpi^3\cO)\supset\cdots$$
    which forms a basis for the topology at the identity.
    This motivates the following definition:
    \begin{definition}
        The \textit{depth} of $\chi$ is the smallest integer $r$ such that $\chi$ is trivial on $T(1+\varpi^{r+1}\cO)$.
    \end{definition}
    In particular, if $\chi$ has depth-zero, then $\chi$ is trivial on $T(1+\varpi\cO)$ and therefore it factors through the quotient $T(\cO)\rightarrow T(k_F)$. We shall abuse notation and denote the resulting character $T(k_F)\rightarrow \CC^{\times}$ also as $\chi$. 

    Moreover, if $I$ is the Iwahori subgroup of $G$ and $I^+$ is its pro-$p$ unipotent radical, then there is an isomorphism
    $$T(\cO)/T(\cO)\cap I^+\cong I/I^+$$
    and therefore $\chi$ determines a character $\rho_\chi$ that is trivial on $I^+$. Explicitly, $I$ has an Iwahori decomposition $$I=(I\cap\bar U)\cdot T(\cO)\cdot(I\cap U),$$
    and $\rho_\chi$ extends $\chi$ in $T(\cO)$ and is trivial on $I\cap\bar U$ and $I\cap U$.

    \begin{theorem}
        Suppose that $\chi$ is a depth-zero character of $T(\cO)$. Then $(I,\rho_\chi)$ is a $[T,\tilde\chi]$-type, where $\tilde\chi$ is any extension of $\chi$ to $T(F)$.
    \end{theorem}

    This theorem motivates us to study the Hecke algebra $\cH(G,I,\rho_\chi)$, where $\rho_\chi$ is a character of $I$ arising from a depth-zero character of $T(\cO)$ as described above. 



\end{document}